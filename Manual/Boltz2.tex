\documentclass[preprint,notoc]{JHEP3}
\usepackage{epsfig}

\def\Vind{V^{\rm induced}}
\def\eslt{\not\!\!{E_T}}
\def\eslt{E_T^{\rm miss}}
\def\emiss{\not\!\!{E}}
\def\to{\rightarrow}
\def\Phat{\hat{\Phi}}
\def\bi{\begin{itemize}}
\def\ei{\end{itemize}}
\def\te{\tilde e}
\def\c1p{C1^\prime}
\def\ta{\tilde a}
\def\tG{\widetilde G}
\def\th{\tilde h}
\def\tH{\tilde H}
\def\tl{\tilde l}
\def\tu{\tilde u}
\def\tc{\tilde c}
\def\ta{\tilde a}
\def\ts{\tilde s}
\def\tb{\tilde b}
\def\tf{\tilde f}
\def\td{\tilde d}
\def\tQ{\tilde Q}
\def\tL{\tilde L}
\def\tH{\tilde H}
\def\tst{\tilde t}
\def\ttau{\tilde \tau}
\def\tmu{\tilde \mu}
\def\tg{\tilde g}
\def\tnu{\tilde\nu}
\def\tell{\tilde\ell}
\def\tq{\tilde q}
\def\tB{\widetilde B}
\def\tw{\widetilde W}
\def\tz{\widetilde Z}
\def\mgut{M_{\rm GUT}}
%\def\tw{\tilde\chi}
%\def\twp{\tilde\chi^+}
%\def\twm{\tilde\chi^-}
%\def\twpm{\tilde\chi^\pm}
%\def\tz{\tilde\chi^0}
%\def\alt{\stackrel{<}{\sim}}
%\def\agt{\stackrel{>}{\sim}}
\def\alt{\lesssim}
\def\agt{\gtrsim}
\def\be{\begin{equation}}  
\def\ee{\end{equation}}  
\def\bea{\begin{eqnarray}}  
\def\eea{\end{eqnarray}}  
\def\CM{\cal M}
\def\sigv{\langle \sigma v \rangle}
\def\To{\Rightarrow}
\def\to{\rightarrow}
\newcommand\drv[2]{\frac{\partial #1}{\partial #2}}
\newcommand\Drv[2]{\frac{d #1}{d #2}}

\title{Boltzmann Equations for the PQMSSM (v2.3)}
\author{Andre Lessa}
\abstract{}


\begin{document}

\section{General Formalism and Approximations}

The general Boltzmann equation for the number distribution of a particle species can be written as\cite{turner} (assuming isotropy):
\be
\drv{F_{i}}{t} -H p \drv{F_{i}}{p} = C_{i}[F_{i},F_{j},p] \label{eq:d1}
\ee
where $F_{i}(p)$ is the number distribution of particle $i$ as function of momentum $p$, $C$ represents a source/sink term and $H$ is the Hubble constant:
\be
H = \sqrt{\frac{8 \pi}{3} \frac{\rho_T}{M_P^2}} \label{H} 
\ee 
with $M_{P} = 1.22\times 10^{19}$ GeV and $\rho_T = \sum_{i} \rho_i$. The number, energy and pressure densities are given in terms
of $F_{i}$ as:
\bea
n_{i}(t) & = & \int \frac{dp}{2 \pi^2} p^2 F_i(p) \nonumber \\ 
\rho_{i}(t) & = & \int \frac{dp}{2 \pi^2} p^2 E_i F_i(p) \label{beqs}\\
P_{i}(t) & = & \frac{1}{3} \int \frac{dp}{2 \pi^2} \frac{p^4}{E_i} F_i(p) \nonumber
\eea
where $m_i$ is the mass of particle $i$ and $E_i = \sqrt{p_i^2 + m_i^2}$. Using Eq.(\ref{eq:d1}), we obtain the following equations for the number and energy densities:
\bea
\Drv{n_i}{t} + 3H n_i & = & \int \frac{dp}{2 \pi^2} p^2 C_i \nonumber \\
\Drv{\rho_i}{t} + 3H (\rho_i + P_i) & = & \int \frac{dp}{2 \pi^2} p^2 E_i C_i \label{eq:meqs}
\eea

The collision term, $C_i$, for the process $i + f + \ldots \leftrightarrow a
+ b + c + \ldots$ is given by\cite{kawasaki0}:
\bea
C_i & = & \frac{1}{E_i} \int \prod_{j,a} \frac{d^3 p_j}{2 E_j (2 \pi)^3}
\frac{d^3 p_a}{2 E_a (2 \pi)^3} (2 \pi)^4 \delta^{4}\left(p_i + p_j + \ldots - p_a - p_b
\ldots\right) |\mathcal{M}|^2 \nonumber \\
&\times& \left[(1 \pm f_a) (1 \pm
f_b)\ldots f_i f_j\ldots - f_a f_b \ldots (1 \pm f_i)(1 \pm f_j)\ldots \right]
\eea
where the plus (minus) sign is for bosons (fermions). Below we always assume
$f_{i,j,a,..} \ll 1$, so:
\bea
C_i & \simeq & \frac{1}{E_i} \int \prod_{j,a} \frac{d^3 p_j}{2 E_j (2 \pi)^3}
\frac{d^3 p_a}{2 E_a (2 \pi)^3} (2 \pi)^4 \delta^{4}\left(p_i + p_j + \ldots - p_a - p_b
\ldots\right) |\mathcal{M}|^2 \nonumber \\
&\times& \left[f_i f_j\ldots - f_a f_b \ldots \right]
\eea

We will assume that $C$ is given by:
\be
C = C_{dec} + C_{prod} + C_{ann} 
\ee
where $C_{dec}$ contains the contributions from decays and inverse decays
($i \leftrightarrow a + b + \ldots$), $C_{prod}$ contains the
contributions from decay injection and inverse decay injection 
($a \leftrightarrow i + b + \ldots$) and $C_{ann}$ from
annihilations with the thermal plasma ($i + i \leftrightarrow a + b$).
Below we compute each term separately, under some assumptions.

\subsection{Annihilation Term}

The annihilation term $C_{ann}$ for the $i + j \leftrightarrow a + b$ process is
given by\cite{turner}:
\be
\int \frac{dp}{2 \pi^2} p^2 C_{ann} = \int d\Pi_{i} d\Pi_{j} d\Pi_{a}
d\Pi_{b} (2 \pi)^4 \delta^{(4)}(p_i + p_j - p_a - p_b) |M|^2 \left[ f_a f_b -
f_i f_j \right]
\ee
where $d\Pi_{i} = d^{3} p_i/((2\pi)^3 2 E_i)$. Since we are ultimately interested in Eqs.(\ref{eq:meqs}) for the number and energy densities, we will
consider the following integral:
\be
\int \frac{dp}{2 \pi^2} p^2 C_{ann}  E_i^{\alpha} = \int d\Pi_{i} d\Pi_{j} d\Pi_{a} d\Pi_{b} (2 \pi)^4 
\delta^{(4)}(p_i + p_j - p_a - p_b) |M|^2
 \left[ f_a f_b - f_i f_j \right] E_i^{\alpha}
\ee
where $\alpha = 0 (1)$ for the number (energy) density. Here we assume that the
distributions can be approximated by\footnote{This approximation is only valid for 
particles with a thermal distribution. However, since the annihilation term
is responsible for keeping the particle $i$ in thermal equilibrium with the plasma, 
it is reasonable to assume a thermal distribution for $i$
while the annihilation term is relevant.}:
\be
f_i \simeq \exp(-(E_i - \mu_i)/T)
\ee
so the annihilation term can then be written as:
\bea
& \int & \frac{dp}{2 \pi^2} p^2 C_{ann}  E_i^{\alpha} =  -\left( \exp((\mu_i + \mu_j)/T) -\exp((\mu_a + \mu_b)/T)\right) \nonumber \\
 & \times & \int  d\Pi_{i} d\Pi_{j} d\Pi_{a} d\Pi_{b} (2 \pi)^4 \delta^{(4)}(p_i + p_j - p_a - p_b) |M|^2 \exp(-(E_i + E_j)/T) \times E_i^{\alpha} \nonumber
\eea
where above we have used conservation of energy ($E_i + E_j = E_a + E_b$). Since for the cases of interest
the equilibrium distributions have zero chemical potential, we have:
\be
\frac{n_i}{\bar{n}_i} = \exp(\mu_i/T)
\ee
so:
\bea
& \int & \frac{dp}{2 \pi^2} p^2 C_{ann} E_i^{\alpha} = -\left( \frac{n_i n_j}{\bar{n}_i \bar{n}_j} - \frac{n_a n_b}{\bar{n}_a \bar{n}_b}\right) \nonumber \\
 & \times & \int  d\Pi_{i} d\Pi_{j} d\Pi_{a} d\Pi_{b} (2 \pi)^4 \delta^{(4)}(p_i + p_j - p_a - p_b) |M|^2 \exp(-(E_i + E_j)/T) \times E_i^{\alpha} \nonumber
\eea
In particular, for the process $i + i \leftrightarrow a + b$, where $a$ and $b$ are in thermal equilibrium ($\mu_a = \mu_b = 0$):
\bea
& \int & \frac{dp}{2 \pi^2} p^2 C_{ann} E_i^{\alpha} =  -\left( \frac{n_i^2}{\bar{n}_i^2} - 1 \right) \nonumber \\
&  \times & \int d\Pi_{i} d\Pi_{j} d\Pi_{a} d\Pi_{b} (2 \pi)^4 \delta^{(4)}(p_i + p_j - p_a - p_b) |M|^2 \exp(-(E_i + E_j)/T) \times E_i^{\alpha}  \nonumber \\
 & = & -\left( n_i^2 - \bar{n}_i^2 \right) \langle \sigma v E_i^{\alpha} \rangle
\eea
For $\alpha = 0$, the above equation is the well known contribution from thermal scatterings to the annihilation term. To estimate its value for $\alpha = 1$, we assume:
\be
\langle \sigma v E \rangle \simeq \langle \sigma v \rangle \langle E_i \rangle = \langle \sigma v \rangle \frac{\rho_i}{n_i} \label{eq:app}
\ee
where $\langle \;\; \rangle$ represents thermal average.
Thus:
\be
\int \frac{dp}{2 \pi^2} p^2 C_{ann} E_i^{\alpha}  = \left( \bar{n}_i^2 - n_i^2 \right) \left\{ \begin{array}{rl}  
\langle \sigma v \rangle & \mbox{, for $\alpha = 0$} \\
\langle \sigma v \rangle \frac{\rho_i}{n_i} &\mbox{, for $\alpha = 1$}
\end{array} \right. \label{eq:collfin}
\ee


\subsection{Decay Term}

Now we derive a simplified expression for the decay (and inverse decay) term,
under approximations similar to the ones used in the last section.
The decay term includes the contributions from particle decay and inverse
decay\cite{kawasaki0,kawasaki}:
\be
C_{dec} \simeq \frac{1}{E_i} \int \prod_{a} \frac{d^3 p_a}{2 E_a (2 \pi)^3}
(2 \pi)^4 \delta^{4}\left(p_i - p_a - p_b \ldots\right) |\mathcal{M}|^2 \left[f_i - f_a f_b \ldots \right]
\label{eq:dec0}
\ee
As in the case of the annihilation term, we assume that the distributions for
$a,b,\ldots$ can be approximated by $f_x \simeq \exp(-(E_x -
\mu_x)/T)$, so we can write:
\be
f_a f_b \ldots \simeq \exp\left(\frac{\mu_a +
\mu_b + \ldots}{T}\right) \exp(-E_i/T) = \frac{n_a n_b \ldots}{\bar{n}_a
\bar{n}_b \ldots} \exp(-E_i/T)  =  \frac{n_a n_b \ldots}{\bar{n}_a \bar{n}_b \ldots}
\bar{f}_{i}
\ee
where we used conservation of energy ($E_a + E_b + \ldots = E_i$) and
$\bar{f}_i$ is the equilibrium distribution for the species $i$.
Hence we can write Eq.(\ref{eq:dec0}) as:
\bea
C_{dec} & \simeq & \left[f_i - \frac{n_a n_b \ldots}{\bar{n}_a \bar{n}_b \ldots}
\bar{f}_{i} \right] \frac{1}{E_i} \int \prod_{a}
\frac{d^3 p_a}{2 E_a (2 \pi)^3} (2 \pi)^4 \delta^{4}\left(p_i - p_a - p_b
\ldots\right) |\mathcal{M}|^2 \nonumber \\
& = & \mathcal{B}_{ab\ldots} \frac{\Gamma_i m_i}{E_i} \left[f_i -
\frac{n_a n_b \ldots}{\bar{n}_a \bar{n}_b \ldots} \bar{f}_{i} \right] 
\eea
where $\Gamma_i$ is the width for $i$ and 
$\mathcal{B}_{ab\ldots} \equiv BR(i \to a + b + \ldots)$

Once again we consider the integral:
\bea
\int \frac{dp}{2 \pi^2} p^2 C_{dec}(p) E_i^{\alpha} = 
 & - & \Gamma_i \int \frac{dp}{2 \pi^2} p^2 \frac{m_i}{E_i} f_i E_i^{\alpha}
 \nonumber \\
 & + & \sum_{i \; decays} \mathcal{B}_{ab\ldots}
\Gamma_i \frac{n_a n_b \ldots}{\bar{n}_a \bar{n}_b \ldots} \int \frac{dp}{2 \pi^2}
p^2 \frac{m_i}{E_i} \bar{f}_{i} E_i^\alpha \label{eq:dec2}
\eea
where we have included the sum over all decay channels and $\alpha = 0 (1)$
for the contribution to the number (energy) density equation.
Note that both integrals are identical, except for the replacement
$f_i \to \bar{f_i}$. The first integral in Eq.(\ref{eq:dec2}) gives:
\be
-\Gamma_i \int \frac{dp}{2 \pi^2} p^2 \frac{m_i}{E_i} f_i(p) E_i^{\alpha} =
\left\{ \begin{array}{rl} -\Gamma_i m_i n_i \langle \frac{1}{E_i} \rangle  & \mbox{, for $\alpha = 0$} \\
-\Gamma_i m_i n_i &\mbox{, for $\alpha = 1$}
\end{array} \right. \label{eq:dec1a}
\ee
where
\be
\langle \frac{1}{E_i} \rangle \equiv \frac{1}{n_i} \int \frac{dp}{2 \pi^2} p^2
\frac{1}{E_i} f_i(p)
\ee
Hence we can write Eq.(\ref{eq:dec2}) as:
\be
\int \frac{dp}{2 \pi^2} p^2 C_{dec}(p) E_i^{\alpha} = -\Gamma_i m_i 
\left\{ \begin{array}{ll} n_i \langle \frac{1}{E_i} \rangle - \bar{n}_i  \langle
\frac{1}{E_i}
\rangle_{eq} \sum \mathcal{B}_{ab\ldots}
 \frac{n_a n_b\ldots}{\bar{n}_a \bar{n}_b\ldots}  & \mbox{, for $\alpha = 0$}  \\
 n_i - \bar{n}_i \sum \mathcal{B}_{ab\ldots}
 \frac{n_a n_b\ldots}{\bar{n}_a \bar{n}_b\ldots}  & \mbox{, for $\alpha = 1$}
\end{array} \right. \label{eq:decfin}
\ee


For the non-equilibrium average we assume:
\be
\langle \frac{1}{E_i} \rangle \simeq \frac{1}{\langle E_i \rangle} =
\frac{n_i}{\rho_i}
\ee
which is exact in the non-relativistic limit, but it is only an approximation
for the relativistic case.
Although we can compute the equilibrium average ($\langle
\frac{1}{E_i}\rangle_{eq}$)
explicitly, in order to have an exact cancellation between 
the decay and inverse decay terms when $i$, $a$ and $b$ are all in equilibrium,
we take:
\be
\langle \frac{1}{E_i} \rangle_{eq} \simeq \langle \frac{1}{E_i} \rangle =
\frac{n_i}{\rho_i}
\ee
With the above approximations we finally obtain:

\be
\int \frac{dp}{2 \pi^2} p^2 C_{dec}(p) E_i^{\alpha} = 
 -\Gamma_i  m_i \left\{ \begin{array}{ll}\frac{n_i}{\rho_i}\left( n_i -
 \bar{n}_i \sum \mathcal{B}_{ab\ldots}
 \frac{n_a n_b \ldots}{\bar{n}_a \bar{n}_b \ldots} \right)   &
 \mbox{, for $\alpha = 0$}
 \\
 n_i - \bar{n}_i \sum \mathcal{B}_{ab\ldots}
 \frac{n_a n_b \ldots}{\bar{n}_a \bar{n}_b \ldots}  & \mbox{, for $\alpha = 1$}
\end{array} \right. \label{eq:decfin}
\ee
where $\mathcal{B}_{ab\ldots} \equiv BR(i\to a+b+\ldots)$.



\subsection{Production Term}

The decay and inverse decay of other particles ($a \to i + b + \ldots$) can also
affect the species $i$. The contribution from these terms we label $C_{prod}$, which is given
by\cite{kawasaki0}:
\be
C_{prod} \simeq \frac{1}{E_i} \int \frac{d^3 p_a}{2 E_a (2
\pi)^3} \prod_{b} \frac{d^3 p_b}{2 E_b (2 \pi)^3} (2 \pi)^4 \delta^{4}\left(p_a
- p_i - p_b \ldots\right) |\mathcal{M}|^2 \left[f_a - f_i f_b \ldots \right]
\ee
Using the same approximations of the previous section, we write:
\be
f_i f_b\ldots \simeq  \frac{n_i n_b \ldots}{\bar{n}_i \bar{n}_b \ldots}
e^{-E_a/T} = \frac{n_i n_b \ldots}{\bar{n}_i \bar{n}_b \ldots}
\bar{f}_{a}
\ee 
Hence:
\be
C_{prod} = \frac{1}{E_i} \int \frac{d^3 p_a}{2 E_a (2 \pi)^3} \prod_{b} \frac{d^3 p_b}{2 E_b (2 \pi)^3} 
(2 \pi)^4 \delta^{4}\left(p_a - p_i - p_b \ldots\right) |\mathcal{M}|^2
\left(f_a - \bar{f}_a \frac{n_i n_b \ldots}{\bar{n}_i
\bar{n}_b \ldots} \right)
\ee
and
\bea
\int \frac{dp}{2 \pi^2} p^2 C_{prod}(p) E_i^\alpha & = & 
\int \frac{d^3 p_a}{E_a (2 \pi)^3} \left(f_a - \bar{f}_a \frac{n_i n_b \ldots}{\bar{n}_i
\bar{n}_b \ldots} \right) \nonumber \\
& \times & \frac{d^3 p E_i^{\alpha}}{2 E_i (2 \pi)^3}
\prod_{b} \frac{d^3 p_b}{2 E_b (2 \pi)^3} (2 \pi)^4 \delta^{4}\left(p_a - p_i - p_b \ldots\right) |\mathcal{M}|^2
\label{eq:prod2}
\eea
with $\alpha = 0 (1)$ for the contribution to the number (energy) density equation.
For $\alpha = 0$ we obtain:
\bea
\int \frac{dp}{2 \pi^2} p^2 C_{prod}(p) & = & \Gamma_a  \mathcal{B}_{i} m_a 
\int \frac{d^3 p_a}{E_a (2 \pi)^3} \left(f_a - \bar{f}_a \sum_b
\frac{\mathcal{B}_{ib\ldots}}{\mathcal{B}_{i}}\frac{n_i n_b \ldots}{\bar{n}_i
\bar{n}_b \ldots} \right)
\nonumber
\\
& = & \Gamma_a \mathcal{B}_{i} m_a \frac{n_a}{\rho_a} \left( n_a - \bar{n}_a
  \sum_b \frac{\mathcal{B}_{ib\ldots}}{\mathcal{B}_{i}} \frac{n_i n_b
  \ldots}{\bar{n}_i \bar{n}_b \ldots} \right)
\eea
where $\mathcal{B}_{ib\ldots} \equiv BR(a \to i + b + \ldots)$, $\mathcal{B}_i
= \sum_{b} \mathcal{B}_{ib\ldots}$ and we have once again assumed $\langle 1/E_a
\rangle \simeq \langle 1/E_a \rangle_{eq} \simeq n_a/\rho_a$.


For $\alpha = 1$, the integral in Eq.(\ref{eq:prod2}) does not take a simple
form. In order to compute it, we assume:
\be
E_i \simeq \frac{E_a}{2}
\ee
The above expression is only exact for 2-body decays and $m_a \gg
m_i,m_b$. For the remaining cases, it is only an estimate.
\bea
\int \frac{dp}{2 \pi^2} p^2 C_{prod}(p) E_i & \simeq & 
\Gamma_a \mathcal{B}_{i}  \frac{m_a}{2} \int \frac{d^3 p_a}{(2
\pi)^3} \left(f_a - \bar{f}_a \sum_b
\frac{\mathcal{B}_{ib\ldots}}{\mathcal{B}_{i}}
 \frac{n_i n_b \ldots}{\bar{n}_i \bar{n}_b \ldots} \right)
\nonumber
\\
& = & \Gamma_a \mathcal{B}_{i}  \frac{m_a}{2} \left( n_a -
\bar{n}_a \sum_b \frac{\mathcal{B}_{ib\ldots}}{\mathcal{B}_{i}} \frac{n_i n_b
\ldots}{\bar{n}_i \bar{n}_b \ldots} \right)
\eea


Combining the results for $\alpha = 0$ and 1, we have:
\be
\int \frac{dp}{2 \pi^2} p^2 C_{prod}(p) E_i^{\alpha} = 
\Gamma_a \mathcal{B}_{i} m_a  \left( n_a - \bar{n}_a
\sum_b \frac{\mathcal{B}_{ib\ldots}}{\mathcal{B}_{i}} \frac{n_i n_b
\ldots}{\bar{n}_i
\bar{n}_b \ldots} \right) \left\{ \begin{array}{ll}  \frac{n_a}{\rho_a}  & \mbox{, for $\alpha = 0$} 
\\
 \frac{1}{2}  & \mbox{, for $\alpha = 1$}
\end{array} \right. \label{eq:prodfin}
\ee

\subsection{Number and Energy Density Equations}

Using the results of Eqs.(\ref{eq:collfin}), (\ref{eq:decfin}) and
(\ref{eq:prodfin}) in the Boltzmann equations for $n_i$ and $\rho_i$
(Eq.(\ref{eq:meqs})), we obtain:
\bea
\Drv{n_i}{t} + 3H n_i  & = &  \left( \bar{n}_i^2 - n_i^2 \right) \langle \sigma
v \rangle - \Gamma_i m_i \frac{n_i}{\rho_i}\left(n_i - \bar{n}_i \sum_{i\to\ldots}
\mathcal{B}_{ab\ldots} \frac{n_a n_b \ldots}{\bar{n}_a \bar{n}_b \ldots} \right)
\nonumber
\\
& + & \sum_a 
\Gamma_a \mathcal{B}_i m_a \frac{n_a}{\rho_a} \left(n_a - \bar{n}_a \sum_{a \to
i\ldots} \frac{\mathcal{B}_{ib\ldots}}{\mathcal{B}_{i}} \frac{n_i n_b \ldots}{\bar{n}_i \bar{n}_b \ldots} \right)  + C_{i}(T) \label{eq:nieq} \\
\Drv{\rho_i}{t} + 3H (\rho_i + P_i) & = & \left( \bar{n}_i^2 - n_i^2 \right)
\langle \sigma v \rangle \frac{\rho_i}{n_i} - \Gamma_i m_i \left( n_i -
\bar{n}_i \sum_{i\to\ldots} \mathcal{B}_{ab\ldots} \frac{n_a n_b\ldots}{\bar{n}_a
\bar{n}_b\ldots}\right) \nonumber \\
 & + & \sum_a \Gamma_a  \mathcal{B}_i \frac{m_a}{2} \left( n_a -
 \bar{n}_a \sum_{a \to i\ldots}  \frac{\mathcal{B}_{ib\ldots}}{\mathcal{B}_{i}} \frac{n_i
 n_b..}{\bar{n}_i \bar{n}_b..} \right) + \tilde{C}_{i}(T)
 \frac{\rho_i}{n_i}
\eea
where $\mathcal{B}_{ab\ldots} = BR(i \to a + b+ \ldots)$, $\mathcal{B}_{ib\ldots} =
BR(a \to i + b + \ldots)$, $\mathcal{B}_i = \sum_b \mathcal{B}_{ib\ldots}$ and
we have included an extra term ($C_i$ and $\tilde{C}_i$) to allow for other possible sources for the number and energy densities. 
For simplicity we assume $C_i = \tilde{C}_{i}$ from now on.


It is also convenient to use the above results to obtain a simpler equation for
$\rho_i/n_i$:
\be
\Drv{\rho_i/n_i}{t} \equiv \Drv{R_i}{t} = -3 H \frac{P_i}{n_i} + \sum_{a}
\mathcal{B}_{i} \frac{\Gamma_a m_a}{n_i} \left( \frac{1}{2} - \frac{n_a}{\rho_a} \frac{\rho_i}{n_i} \right) \left(n_a -
\bar{n}_a \sum_{a \to i\ldots} \frac{\mathcal{B}_{ib\ldots}}{\mathcal{B}_{i}} \frac{n_i
 n_b..}{\bar{n}_i \bar{n}_b..}\right) \label{eq:Rieq}
\ee

Besides the above equations, it is useful to consider the evolution equation for entropy:
\be
dS \equiv \frac{dQ^{dec}}{T}
\ee
where $dQ^{dec}$ is the net energy injected from decays.
With the above definition we have:
\bea
\dot{S} & = & \frac{1}{T}\sum_i BR(i,X)
\frac{d\left(R^3 \rho_i\right)^{dec}}{dt}  \nonumber \\
\To \dot{S} & = & \frac{R^3}{T}\sum_i BR(i,X)
\Gamma_i m_i\left(n_i - \bar{n}_i \sum_{i\to\ldots} \mathcal{B}_{ab\ldots} \frac{n_a n_b\ldots}{\bar{n}_a
\bar{n}_b\ldots} \right) \label{Seq}
\eea
where $R$ is the scale factor and $BR(i,X)$ is the fraction of energy injected
in the thermal bath from $i$ decays.
 
 
The above expressions can be written in a more compact form if we define
the following ''effective thermal densities'' and ''effective BR'':
\bea
\mathcal{N}^{th}_{X} & \equiv &  \bar{n}_X \sum_{X \to \ldots} BR(X \to 1 + 2 +
\ldots)
\prod_{k}
\frac{n_k}{\bar{n}_k} \nonumber \\
\mathcal{N}^{th}_{XY} & \equiv & \frac{\bar{n}_X}{\mathcal{B}^{eff}_{XY}}
\sum_{X \to Y + \ldots} g_Y BR(X \to g_Y Y + 1 + \ldots)
\left(\frac{n_Y}{\bar{n}_Y}\right)^{g_Y} \prod_{k} \frac{n_k}{\bar{n}_k}
\nonumber \\
\mathcal{B}^{eff}_{XY} & \equiv & \sum_{X \to Y + \ldots} g_Y BR(X \to g_Y Y +
1+\ldots) \nonumber
\eea
where $g_Y$ is the $Y$ multiplicity in the final state of $X$ decays. 
In addition, defining:
\be
x = \ln(R/R_0),\;\; N_i = \ln(n_i/s_0),\;\; {\rm and}\;\; N_S = \ln(S/S_0)
\ee
we can write Eqs.(\ref{Seq}), (\ref{eq:nieq}) and (\ref{eq:Rieq}) as:
\bea
N_S' & = & \frac{e^{(3 x - N_S)}}{HT} \sum_{i} BR(i,X) \Gamma_i m_i \left(n_i -
\mathcal{N}_{i}^{th} \right) 
\label{Seqb} \\
N_i' & = & -3 + \frac{\sigv_i}{H} n_i [\left(\frac{\bar{n}_i}{n_i}\right)^2
-1] -  \frac{\Gamma_i}{H} \frac{m_i}{R_i}\left(1 -
\frac{\mathcal{N}_{i}^{th}}{n_i} \right) \nonumber  \\
 & + & \sum_{a} \mathcal{B}_{ai}^{eff} \frac{\Gamma_a}{H}
 \frac{m_a}{R_a}\left(\frac{n_a}{n_i} - \frac{\mathcal{N}_{ai}^{th}}{n_i}
  \right)
 \\
R_i' & = &  -3 \frac{P_i}{n_i} + \sum_{a} \mathcal{B}_{ai}^{eff}
\frac{\Gamma_a}{H} m_a \left( \frac{1}{2} - \frac{R_i}{R_a} \right) \left(\frac{n_a}{n_i} -
\frac{\mathcal{N}_{ai}^{th}}{n_i} \right)
\label{Nieq}
\eea
where $'=d/dx$.

The above equation for $N_i$ also applies for coherent oscillating fields, if we define:
\be
N_i = \ln(n_i/s_0),\;\; {\rm and}\;\; n_i \equiv \rho_i/m_i
\ee
so
\bea
N_i' & = & -3 - \frac{\Gamma_i}{H}  \nonumber \\
R_i'& = & 0 \label{Nico}
\eea
where we assume that the coherent oscillating component does not couple to any of the other fields.

Collecting Eqs.(\ref{Seqb})-(\ref{Nieq}) and (\ref{Nico}) we have a closed set of first order differential equations:
\bi
\item Entropy:
\be
N_S' = \frac{e^{(3 x - N_S)}}{HT} \sum_{i} BR(i,X) \Gamma_i m_i \left(n_i -
\mathcal{N}_{i}^{th} \right) \label{eq:Sfin}
\ee
\item Thermal fields:
\bea
N_i'& = & -3 + \frac{\sigv_i}{H} n_i [\left(\frac{\bar{n}_i}{n_i}\right)^2
-1] -  \frac{\Gamma_i}{H} \frac{m_i}{R_i}\left(1 - \frac{\mathcal{N}_{i}^{th}}{n_i}
 \right)
  +  \sum_{a} \mathcal{B}_{ai}^{eff} \frac{\Gamma_a}{H}
 \frac{m_a}{R_a}\left(\frac{n_a}{n_i} - \frac{\mathcal{N}_{ai}^{th}}{n_i}
  \right) \nonumber
 \\
R_i' & = &  -3 \frac{P_i}{n_i} + \sum_{a} \mathcal{B}_{ai}^{eff}
\frac{\Gamma_a}{H} m_a \left( \frac{1}{2} - \frac{R_i}{R_a} \right) \left(\frac{n_a}{n_i} -
\frac{\mathcal{N}_{ai}^{th}}{n_i} \right)
\eea
\item Coherent Oscillating fields:
\bea
N_i' & = & -3 - \frac{\Gamma_i}{H} \nonumber \\
R_i' & = & 0 \label{eq:COeq}
\eea
\ei


As seen above, the equation for $R_i = \rho_i/n_i$ depends on $P_i/n_i$. A proper evaluation of this quantity
requires knowledge of the distribution $F_i(p,t)$. However, for relativistic (or
massless) particles we have $P_i = \rho_i/3$, as seen from Eq.(\ref{beqs}), while for particles at rest we have $P_i = 0$. Hence $F_i(p,t)$ is only 
required to evaluate the relativistic/non-relativistic transition, which corresponds to a relatively small part of the evolution history
of particle $i$. Nonetheless, to model this transition we approximate $F_i$ by a thermal distribution and take $T_i, \mu_i \ll m_i$, where $T_i$ is
the temperature of the particle (which can be different from the thermal bath's). Under these approximations we have:
\bea
\frac{P_i}{n_i} & = & T_i \nonumber \\
\frac{\rho_i}{n_i} & = & T_i \left[ \frac{K_1(m_i/T_i)}{K_2(m_i/T_i)} \frac{m_i}{T_i} + 3 \right] \label{eq:p1}
\eea
where $K_{1,2}$ are the modified Bessel functions. In particular, if $m_i/T_i \gg 1$:
\be
\frac{\rho_i}{n_i} \simeq T_i \left[\frac{3}{2} + \frac{m_i}{T_i}  + 3 \right] \To \frac{P_i}{n_i} = T_i = \frac{2 m_i}{3}\left( \frac{R_i}{m_i} -1 \right)
\ee
As shown above, for a given value of $R_i = \rho_i/n_i$, Eq.(\ref{eq:p1}) can be inverted to compute $T_i$ ($=P_i/n_i$):
\be
\frac{P_i}{n_i} = T_i(R_i)
\ee
Since we are interested in the non-relativistic/relativistic transition, we can expand the above expression around $R_i/m_i = 1$,
so $P_i/n_i$ can be written as:
\be
\frac{P_i}{n_i} = \frac{2 m_i}{3}\left( \frac{R_i}{m_i} -1 \right) + m_i \sum_{n >1} a_n \left(\frac{R_i}{m_i} -1 \right)^n
\ee
where the coefficients $a_n$ can be numerically computed from Eq.(\ref{eq:p1}). The above approximation should be valid for
$m_i/T_i \gtrsim 1$ (or $R_i \gtrsim m_i$). On the other hand, for $m_i/T_i \ll 1$ (or $R_i \gg m_i$), we have the 
relativistic regime, with $P_i/n_i = R_i/3$.
Therefore we can approximate the $P_i/n_i$ function for all values of $R_i$ by:
\be
\frac{P_i}{n_i} = \left\{ \begin{array}{rl}
& \frac{2 m_i}{3}\left( \frac{R_i}{m_i} -1 \right) + m_i \sum_{n >1} a_n \left(\frac{R_i}{m_i} -1 \right)^n  \mbox{ , for $R_i < \tilde{R}$} \\
& \frac{R_i}{3}  \mbox{ , for $R_i > \tilde{R}$} 
\end{array} \right. \label{Pfin}
\ee
where the coefficients $a_n$ are given by the numerical fit of Eq.(\ref{eq:p1}) and $\tilde{R}$ is given by the matching of the two solutions.

Finally, to solve Eqs.(\ref{eq:Sfin})-(\ref{eq:COeq}) we need to compute $H$
according to Eq.(\ref{H}), which requires knowledge of the energy densities for all
particles ($\rho_i$) and for the thermal bath ($\rho_R$). The former are directly obtained from $N_i$ and $R_i$, while the latter can be
 computed from $N_S$:
\be
T = \left(\frac{g_{*S}(T_R)}{g_{*S}(T)}\right)^{1/3} T_R \exp[N_S/3 -x] \To \rho_R = \frac{\pi^2}{30} g_{*}(T) T^4
\ee

Eqs.(\ref{eq:Sfin})-(\ref{eq:COeq}), with the auxiliary equations for $H$
(Eq.(\ref{H})) and $P_i/n_i$ (Eq.(\ref{Pfin})) form a set of closed equations, which can be solved once the initial conditions for the number
density ($n_i$), energy density ($\rho_i$) and entropy ($S$) are given. For thermal fluids we assume:
\bea
n_i(T_R) & = & \left\{ 
\begin{array}{ll} 
0 & , \mbox{ if $\sigv_i \bar{n}_i/H|_{T=T_R} < 10$} \\
\bar{n}_i(T_R) & , \mbox{ if $\sigv_i \bar{n}_i/H|_{T=T_R} > 10$} 
\end{array} \right. \label{ni0TP} \\
\frac{\rho_i}{n_i}(T_R) & = & \frac{\bar{\rho}_i}{\bar{n}_i}(T_R)
\eea
where $\bar{\rho}_i$ is the equilibrium energy density (with zero chemical potential) for the particle $i$.
While for coherent oscillating fluids the initial condition is set at the beginning of oscillations:
\bea
n_i(T^{osc}_i) & = &\frac{\rho_i^{0}}{m_i(T^{osc}_i)} \\
\frac{\rho_i}{n_i}(T^{osc}_i) & = & m_i
\eea
where $T^{osc}_i$ is the oscillation temperature, given by $3H(T^{osc}_i) = m_i(T^{osc}_i)$ and $\rho_i^{0}$ the
initial energy density for oscillations.

Finally, the initial condition for the entropy $S$ is trivially obtained, once we assume a radiation dominated universe
at $T=T_R$:
\be
S(T_R) = \frac{2 \pi^2}{45} g_*(T_R) T_R^3 R_0^3
\ee



\section{Code}

Here we describe how the above formalism is implemented in a numerical code for solving the coupled Boltzmann equations.
In Sec.\ref{sec:In} we describe how the input for specific models should be defined. Then, in Sec.\ref{sec:Main} we
outline the procedure used to solve the Boltzmann equations and to treat some of the discrete transitions
required by the formalism described above. Finally, in Sec.\ref{sec:Out} we describe what is the output of the code and
how it can be controlled by the user.

\subsection{Input}
\label{sec:In}

In order to solve the Boltzmann equations for a particular model,
the user has to provide the SUBROUTINE INPUTBOLTZ(T), which, for a given (thermal bath) temperature $T$,
fills the COMMON BLOCK:
\begin{center}
	COMMON/INBOLTZ/BR(NP,NP),DEGF(NP),MASS(NP),GAM(NP),\\SIGV(NP),C(NP),COHOSC(NP),TRH,NCOMPS,LABEL(NP)
\end{center}
where NP = 20 and
\bi
\item NCOMPS ($\leq 10$) = the number of particles (the first component must be radiation)
\item MASS(i) = mass for particle i (can be temperature dependent, as in the axion case)
\item DEGF(i) = +-number of degrees of freedom for particle i. A plus sign should be used
for bosons, while a minus should be used for fermions, i.e. DEGF=-2 for neutralinos and DEGF=1 for axions. The
value for the i=1 component (radiation) is never used, since the number of degrees of freedom in this case
in given by the function GSTAR(T).
\item GAM(i) = decay width for particle i, in its rest frame.
\item BR(i,j) = branching ratio for the decay $i \to j + X$, including the multiplicity factor, if the
i particle decays into multiple j's.
\item BR(i,1) = fraction of energy per i particle injected in the radiation fluid.
\item SIGV(i) = thermal averaged cross-section for the annihilation of i particles, as defined in the previous section.
\item C(i) = additional source term for particle i, as defined in the previous section.
\item COHOSC(i) = initial energy density for coherent oscillating particles. Must be zero for thermal (non-oscillating) components.
\item TRH = re-heat temperature.
\item LABEL(i) = label for particle i (optional)
\ei


\subsection{Main Code}
\label{sec:Main}


Once the INPUTBOLTZ subroutine is provided, the user can compute the solution for the Boltzmann equations from T=TRH to T=TF,
calling:
\begin{flushleft}
CALL INPUTBOLTZ(TRH)    ! (iniatilization)\\
CALL EQSBOLTZ(TF,IOUT)  ! (compute solution)
\end{flushleft}
where, if IOUT$>0$, the scale factor ($R$) and energy densities as a function of $T$ are written to UNIT=IOUT. If TF = 0,
the evolution proceeds until all unstable particles have decayed and/or all coherent oscillating components have 
started to oscillate. Before calling EQSBOLTZ, the user must define the parameters which regulate the precision
of the procedure, given by the BLOCK DEPARS:
\begin{center}
COMMON/DEPARS/EPS,DX0,STEP,IERROR
\end{center}
where EPS is the relative precision for the solution $N_i(TF)$, DX0 is the $x$ interval for printing the solutions in IOUT and
the maximum $\Delta x$ step and STEP is the initial $\Delta x$ step for the evolution.
Failure to solve the equations (most likely due to numerical instabilities) is indicated by IERROR$<0$.

Specific components can be turned off using the COMMON BLOCK:
\begin{center}
COMMON/SWITCHES/TURNOFF(NP)
\end{center}
If TURNOFF(I)=.TRUE., the $i-$component will not be included in the evolution of the Boltzmann equations.


The EQSBOLTZ is the main subroutine used to solve the equations, once the appropriate input has been defined. Its main steps
are:
\begin{enumerate}
\item Set initial conditions at $T=TRH$: check which thermal particles are coupled/decoupled to the thermal bath and if coherent
oscillating fluids are already oscillating at T=TRH. Then it sets the initial
number densities and temperatures for each component, as defined in the previous section. Set X1=1 and X2=X1+DX0.
\item Solve the equations between X1 and X2.
\item Check if a particle has decayed. If the particle $i$ satisfies
\be
\Gamma_i/H > 100\;\; {\rm and}\;\; \min_{j \neq i}(\rho_i/\rho_j)< 10^{-3}
\ee
the particle is neglected from here on. 
The decay temperature ($T_D$) is defined by the sudden decay approximation: $\Gamma_i/\gamma_i = H(T_D)$, where $\gamma_i$ is
the boost factor ($\gamma_i \equiv \langle E_i \rangle/m_i$). Although this temperature
is printed out in the output, it is never used in the code.
\item Check if a particle has decoupled from the thermal bath or started to oscillate in the interval (X1,X2).
Decoupling is assumed if $\sigv_i \bar{n}_i < H/10$, which also defines the freeze-out temperature.
The oscillation temperature is given by $3 H(T_{osc}) = m_i(T_{osc})$ and defines the beginning of evolution for
the oscillating components.
\item If a component has started to oscillate or if it has decoupled, loop over this interval with smaller steps until the decoupling or oscillation temperature converges ($\Delta T_i/T_i < 0.1$).
\item Write temperature, scale factor and energy densities to IOUT, if IOUT$>0$.
\item Set X1=X2 and X2=X1+DX0 and return to point 2. until $T<TF$ (or all unstable particles have decayed and all oscillating
fluids have oscillated, if TF$\leq 0$).
\end{enumerate}




\subsection{Output}
\label{sec:Out}

The standard information printed after solving the Boltzmann equations gives:
\bi
\item Freeze-out temperatures ($T_{fr}$) for each thermal component. As mentioned in the last section, $T_{fr}$ is
given by the decoupling condition: $\sigv_i \bar{n}_i = H/10$. Since the decoupling is a continuous process, $T_{fr}$
is just an estimate for the decoupling temperature.
\item Decay temperatures ($T_D$) for each thermal component. Once again the decay process is continuous and $T_D$ given in
the print out is estimated by the sudden decay approximation ($\Gamma_i/\gamma_i = H(T_D)$).
\item Oscillation temperature ($T_{osc}$).
\item Entropy ratio ($S/S_0$). In case of entropy injection from decays of unstable particles, $S/S_0 > 1$.
\item Relic densities ($\Omega_i h^2$) at $T_0 = 2.725$ K. In order to consistently compute the relic densities today,
we evolve $R_i = \rho_i/n_i$ from $TF$ to $T_0$ assuming a trivial universe expansion:
\be
R_i' = - 3 \frac{P_i}{n_i}
\ee
Note that the result obtained above is insensitive to $H$, so it does not matter if there is a transition from
a radiation dominated to a matter dominated (or dark energy dominated) universe between $TF$ and $T_0$. Once $R_i(T_0)$
is obtained, the relic density is given by:
\be
\Omega_i h^2 = n_i(TF) \times \frac{g_{*S}(T_0) T_0^3}{g_{*S}(TF) TF^3} \times \frac{R_i(T_0)}{\rho_c/h^2}
\ee
\item Relic densities before decay. It may be relevant to compute the relic densities of an unstable particle as it would be given
if it had not decayed. In particular, this value can be used to impose BBN bounds on the decays. This quantity is computed after
the particle becomes non-relativistic and well before the decay starts ($\Gamma_i/H(T) = 1/10$)  and is given by:
\be
\tilde{\Omega}_i h^2 = \frac{\rho_i(T)}{s(T)} \times \frac{s(T_0)}{\rho_c/h^2}
\ee
Note that the above expression assumes a radiation dominated universe from $T$ to $T_0$ and should be used with caution.
\item Effective number of (new) neutrinos ($\Delta N_{eff}$). Since neutrinos are still coupled for $T > 1$ MeV, this quantity is only compute below
this temperature. $\Delta N_{eff}$ is given by:
\be
\Delta N_{eff}(T) = \frac{\rho_{DR}(T)}{\rho_{\nu}}
\ee
where $\rho_{DR}$ is the total energy density of relativistic particles (excluding radiation and neutrinos) and $\rho_{\nu}$ is the energy density
of neutrinos after they freeze-out:
\be
\rho_{DR} = \sum_{R_i/m_i > 2} \rho_i \mbox{ and } \rho_{\nu} = \frac{\pi^2}{15}\frac{7}{8}\left(\frac{4}{11}\right)^{4/3} T^4
\ee
Note that $\Delta N_{eff}$ is in general a function of temperature, since $\rho_{DR}$ will decrease if massive particles become 
non-relativistic below 1 MeV.

\ei


Furthermore, if $IOUT>0$, the scale factor ($R$), the energy densities and  $\Delta N_{eff}$ are printed as a function of $T$ in UNIT=IOUT.
Also, the following quantities are stored in COMMON BLOCKs:
\bi
\item Final relic densities and entropy ratio:
\begin{center}
COMMON/OUTPUT/OMEGA(NP),RS
\end{center}
\item Decoupling, oscillation and decay temperatures:
\begin{center}
COMMON/TEMPS/TDEC(NP),TOSC(NP),TDCAY(NP)
\end{center}
\item Relic density of unstable particles before decay ($\tilde{\Omega} h^2$), temperature at the end of entropy injection (if any) and
effective number of new neutrinos ($\Delta N_{eff}$) {\it at the final temperature $TF$}:
\begin{center}
COMMON/BBNINFO/UMEGA(NP),TSTAB,DNeff
\end{center}
\ei

\section{PQMSSM}
In order to apply the above formalism to the PQMSSM we need to define:
\bi
\item Masses: $m_{\tz_1}$, $m_{\ta}$, $m_s$, $m_{a}(T)$, $m_{\tG}$
\item Decay Width: $\Gamma_i$, with $i = \tz_1,\ta,s,\tG$,
\item Branching Ratios: $BR(i,j)$ and $BR(i,1)$, with $i,j = \tz_1,\ta,s,\tG$,
\item Annihilation cross-sections: $\sigv_i$, with $i = a,\tz_1,\ta,s,\tG$,
\item Additional Source terms: $C_i$, with $i = a,\tz_1,\ta,s,\tG$,
\item Initial energy density for coherent oscillating fields: $\rho_i^0$, with $i=a,s$
\item and remaining SUSY spectrum (for computation of $g_*$ and gravitino/axino decays)
\ei

Below we describe how the quantities $\sigv_i$, $\Gamma_i$, $BR(i,j)$, $C_i$ and $\rho_i^0$ are computed.

\subsection{$\sigv$}
The annihilation cross-section for the neutralino component is computed in the SUBROUTINE ZSIG(T,MZ),
where T is the temperature and MZ is the neutralino mass. For efficiency purposes the calculation of
$\sigv_{\tz_1}$ can be controlled through the COMMON BLOCK:
\begin{center}
      COMMON/INSIGMA/INOMGZ,INMZ,INFLAG,INDATA
\end{center}
The main options are set by INFLAG:
\bi
\item If INFLAG=166, the subroutine returns a constant value for $\sigv_{\tz_1}$, given by:
\be
\sigv_{\tz_1} = 1.7\times 10^{-10}/\Omega_{\tz_1} h^2
\ee
where the value for $\Omega_{\tz_1} h^2$ should be set in INOMGZ.
\item If INFLAG=266, then
\bi
\item If INDATA $<0$: generate file with $\sigv_{\tz_1}$ values for $3\times 10^{-5} < T/m_{\tz_1} < 2$ in 
UNIT=ABS(INDATA) for future extrapolation and set INDATA = ABS(INDATA). The number of points and specific values
of $T$ are chosen as to properly describe the shape of $\sigv_{\tz_1}$. The file must be open by the
main program.
\item If INDATA $>0$: use values in UNIT=INDATA for a linear extrapolation in $\log(T/m_{\tz_1})$
\ei
\item If INFLAG $\neq$ 166 and 266: compute $\sigv_{\tz_1}(T)$ (full integration). Due to convergence issues,
$\sigv_{\tz_1}(T > m_{\tz_1}/2) \equiv \sigv_{\tz_1}(T=m_{\tz_1}/2)$ and $\sigv_{\tz_1}(T < m_{\tz_1}/5\times 10^{-5}) \equiv \sigv_{\tz_1}(T=m_{\tz_1}/5\times 10^{-5})$.
\ei

The thermal production rate for axions, saxions, axinos and gravitinos was computed in Refs.\cite{graff2,strumia,pradler}.
However, the expressions derived in Refs.\cite{graff2,strumia,pradler} are only valid for out of equilibrium production, where the
(out of equilibrium) thermal production rate ($W_i$) is defined by:
\be
\frac{d n_i}{dt} + 3 H = W_i \label{eq:prodr}
\ee
In the out of equilibrium regime we have $\bar{n}_i \gg n_i$, so Eq.(\ref{eq:nieq}) becomes:
\be
\frac{d n_i}{dt} + 3 H = \bar{n}_i^2 \sigv_i
\ee
Comparing the above equation to Eq.(\ref{eq:prodr}), we identify:
\be
\sigv_i = \frac{W_i}{\bar{n}_i^2} \label{eq:sigw}
\ee
Although the above relation is only exact for the out of equilibrium regime, we use it for all values of temperature.
This is only a poor approximation if the reheat temperature ($T_R$) is very close to the decoupling temperature ($T_{dec}$).
Since for $T_R \gtrsim T_{dec}$,  Eq.(\ref{eq:nieq}) gives $n_i = \bar{n}_i$ (independent of $\sigv_i$), while for $T_R \lesssim T_{dec}$ Eq.(\ref{eq:sigw}) is exact.


Using the expressions \cite{graff2,strumia,pradler} and Eq.(\ref{eq:sigw}), we obtain\footnote{According to Ref.\cite{graff2}, the axion and
saxion thermal rates are identical in supersymmetric models.}:
\begin{eqnarray*}
\sigv_{a} & = & \frac{9}{128 \pi^3 \xi(3)} \frac{g_s^6}{f_a^2}\ln(\frac{1.0126}{g_s})\\
\sigv_{s} & = & \sigv_{a}\\
\sigv_{\ta} & = & \frac{1}{576 \pi^3 \xi(3)^2} \frac{g_s^4}{f_a^2}F(g_s)\\
\sigv_{\tG} & = & \frac{1.37}{M_P^2}\times \left[ 72 g_s^2 \ln(1.271/g_s)(1+\frac{M_3^2}{3 m_{\tG}^2})\right. \\
&+& \left. 27 g^2 \ln(1.312/g)(1+\frac{M_2^2}{3 m_{\tG}^2}) + 11 g'^2 \ln(1.266/g')(1+\frac{M_1^2}{3 m_{\tG}^2}) \right]
\end{eqnarray*}
where $F(g_s)$ is numerically obatined from Ref.\cite{strumia}:
\be
F(x) = -0.365771 + 9.38897 x + 27.7315 x^2 - 20.1012 x^3 + 5.153 x^4
\ee

We note that while the expression for $\sigv_{\ta}$ includes thermal decays such as $g \to \tg + \ta$ and is valid for all values of $g_s$,
the expressions for $\sigv_{a,s}$ correspond only to the hard themal loop calculation and assume $g_s \ll 1$ (or $T_R \gg 10^6$ GeV). Nonetheless
we use these expressions for all values of $g_s$ ($T_R$).

\subsection{$\Gamma$,$BR$}
The decay rates are computed through the SUBROUTINES:
\begin{center}
AXINOBR(MAXINO,FA,AXLT,AXWD,AXVIS)\\
GRAVITINOBR(MGT,GLT,GWD)\\
Z1BRS(MZ1,FA,Z1B,Z1WD,Z1LT)\\
SAXIONBR(MSAXION,FA,XI,SAXWD,SAXLT)
\end{center}
where the axino, gravitino, neutralino and saxion decay rates are given by AXWD, GWD, Z1WD and SAXWD, respectively.
Neutralinos and axinos decays into gravitinos (if kinematically allowed) are NOT included. So the gravitino LSP
case is NOT presently included. For the case of an axino LSP, Z1BRS assumes MAXINO=0. Therefore, if $m_{\tz_1} < m_{\ta}$,
the user should set Z1WD=0.

The branching ratios are given by:
\begin{eqnarray*}
BR(s \to X) & = &  1 - BR(s \to aa) - BR(s \to \ta\ta)\\
BR(s \to \tilde{Z}_1) & = & 2\times BR(s \to \tg\tg) \\
BR(s \to a) & = & 2\times BR(s \to aa)
\end{eqnarray*}
\bi
\item If $m_{\tz_1} > m_{\ta}$:
\begin{eqnarray*}
BR(\tz_1 \to \ta) & = &  1 \\
BR(\tz_1 \to X) & = &  1
\end{eqnarray*}
\item If $m_{\ta} > m_{\tz_1}$:
\begin{eqnarray*}
BR(\ta \to \tz_1) & = &  1 \\
BR(\ta \to X) & = &  1 \\
\end{eqnarray*}
\ei
\bi
\item If $m_{\tG} > m_{\tz_1}$:
\begin{eqnarray*}
BR(\tG \to \tz_1) & = &  1 \\
BR(\tG \to X) & = &  1
\end{eqnarray*}
\item If $m_{\tG} < m_{\tz_1}$ (but $m_{\tG} > m_{\ta}$, since $\tG$ LSP is not allowed!):
\begin{eqnarray*}
BR(\tG \to \ta) & = &  1 \\
BR(\tG \to a) & = &  1 
\end{eqnarray*}
\ei
and all other BRs are zero.

As discussed above, the branching ratio $BR(i \to X)$ is defined as the fracion of energy injected in the
thermal bath. For simplicity we assume that most of the energy from decays goes into radiation, so $BR(i \to X) \sim 1$,
except for saxion decays to axions and axinos and gravitino decays to axion + axino.

\subsection{$C$}
The source term $C_i$ can be used to include other processes that can not be described as decays or annihilations.
However these are not relevant for the PQMSSM. Therefore we set all these to zero.

\subsection{$\rho^0$}
Finally, the initial energy densities for the oscillating saxion and axion fields are given by:
\begin{eqnarray*}
\rho^0_a & = & 1.44  \frac{m_a(T)^2 f_a^2 \theta_i^2 }{2} f(\theta_i)^{7/6}\\
\rho^0_s & = & \min\left[1.9 \times 10^{-8}\left(\frac{2\pi^2 g_*(T_R) T_R^3}{45}\right)\left(\frac{T_R}{10^5}\right)\left(\frac{s_i}{10^{12}}\right)^2,\frac{m_s^2 s_i^2}{2}\right]	
\end{eqnarray*}
where $f(\theta_i) = \ln[e/(1-\theta_i^2/\pi^2)]$ and $f_a \theta_i$ and $s_i$ are the initial axion and saxion amplitudes. 
The definition of $\rho^0_s$ accounts for the possibility of beginning of saxion oscillations during inflation ($T_R < T_{osc}$).

% ---- Bibliography ----
%
\begin{thebibliography}{99}
%
\bibitem{turner} E. Kolb and M. Turner, {\it The Early Universe}, Addison-Wesley Pub. (1990).
%
\bibitem{kawasaki0} M. Kawasaki, G. Steigman and H. Kang, {\it Cosmological evolution of generic early-decaying
particles and their daughters}, Nucl.Phys. B402, 323-348 (1993).
%
\bibitem{kawasaki} M. Kawasaki, G. Steigman and H. Kang, {\it Cosmological evolution of an early-decaying particle}, Nucl.Phys. B403, 671-706 (1993).
%
\bibitem{graff} P. Graff and F. Steffen, {\it Thermal axion production in the primordial quark-gluon plasma}, Phys.Rev. D 83, 070511 (2011).
%
\bibitem{graff2} P. Graff and F. Steffen, {\it Axions and saxions from the primordial supersymmetric plasma and extra radiation signatures}, arXiv:1208.2951 (2012).
%
\bibitem{strumia} A. Strumia, {\it Thermal production of axino Dark Matter}, JHEP 06, 036 (2010).
%
\bibitem{pradler} J. Pradler and F. Steffen, {\it Constraints on the reheating temperature in gravitino dark matter scenarios}, Phys.Lett. {\bf B 648}, 224 (2007).
%
\end{thebibliography}


\end{document}
